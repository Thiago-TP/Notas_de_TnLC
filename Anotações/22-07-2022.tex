\section*{22/07/2022 --- Introdução a saídas planas}
\markboth{Introdução a saídas planas}{22/07/2022}
\noindent\textbf{\sffamily Exercício 1.}\\
	Dado uma trajetória polinomial $x(t)$ tal que 
	%
	\[
		x(0) = \dot{x}(0) = \ddot{x}(0) 
		     = \dot{x}(10)= \ddot{x}(10)
		\quad\text{ e }\quad
		x(10) = 1,
	\]
	% 
	encontre $u$ que satisfaça 
	%
	\[
		\ddot{x} = -\sin(x-u).
	\]
	%
	\rule{\textwidth}{0.5pt}\\

\noindent\textbf{\sffamily Exercício 1 --- solução.}\\
	Exitem $6$ equações acerca de $x(t)$, de foma que este é polinômio de ordem $5$ ou maior. 
	Tomemos $x$ de grau 5 genérico, i.e., com $c_i$, $0\leq i\leq 5$, tais que
	%
	\[
		x(t) = c_0 + c_1t + c_2t^2 + c_3t^3 + c_4t^4 + c_5t^5
	\]
	%
	satisfaça às equações no enunciado. Substituindo, temos
	%
	\begin{align*}
		\left\{
			\begin{array}{rl}
				0 &= x(0)         \\ 
				0 &= \dot{x}(0)   \\ 
				0 &= \ddot{x}(0)  \\ 
				0 &= \dot{x}(10)  \\ 
				0 &= \ddot{x}(10) \\ 
				1 &= x(10)     
			\end{array}
		\right.
		\implies
		\left\{
			\begin{array}{rl}
				0 &= c_0 \\ 
				0 &= c_1 \\ 
				0 &= c_2 \\ 
				0 &= 3\cdot 10^2c_3 + 4\cdot 10^3 c_4 + 5\cdot 10^4 c_5 \\ 
				0 &= 6\cdot 10c_3 + 12\cdot 10^2 c_4 + 20\cdot 10^3 c_5 \\ 
				1 &= 10^3c_3 + 10^4c_4 + 10^5c_5 
			\end{array}
		\right.
		\implies
		\begin{pmatrix} 
			c_0 \\ c_1 \\ c_2 \\ c_3 \\ c_4 \\ c_5
		\end{pmatrix}
		= 
		\begin{pmatrix} 
			0 \\ 0 \\ 0 \\ 1/100 \\ -3/2000 \\ 3/50000
		\end{pmatrix}.
	\end{align*}
	%
	Conclui-se então que
	%
	\[
		u(t) = \arcsin(\ddot{x}(t)) + x(t)
		     = \arcsin\left(
		     	   \frac{6}{100}t - \frac{36}{2000}t^2 + \frac{60}{50000}t^3 
		       \right)
		       +
		       \frac{1}{100}t^3 - \frac{3}{2000}t^4 +\frac{3}{50000}t^5.
	\]
	%
	\begin{figure}[H]\centering
		\includegraphics{trajetória x.pdf}
		\caption{Gráfico em escala de $x(t)$ para $0\leq t\leq1$.}
	\end{figure}


